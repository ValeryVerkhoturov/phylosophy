% !TeX encoding = UTF-8
% !TeX spellcheck = russian-aot
% !TeX program = xelatex
\documentclass[14pt]{extarticle}

\usepackage[english, main=russian]{babel}
\usepackage{fontspec}
\setmainfont{Times New Roman}

\usepackage{microtype}


\usepackage{authblk}
\title{Философия.~Доклад.~Учения о~бытии и~познании Платона и~Аристотеля.~Их основные отличия\footnote{Источник~--- Философия: учебник / Е.\,А.\,Никитина, О.\,Г.\,Арапов, Э.\,А.\,Арапова и~(др).
		Под общ. ред. д-ра филос. наук Е.\,А.\,Никитиной.~--- Москва: МИРЭА~--- Российский технологический университет, 2022.~--- 210~с.
		ISBN 978-5-7339-1615-6}}
\author{В.\,С.\,Верхотуров \\ БСБО-05-20}
\affil{РТУ МИРЭА}

\begin{document}
	\maketitle
	
	
	\section{Учения о~бытии и~познании Платона}
	Платон (427--347 до~н.~э.)~--- величайший мыслитель античной эпохи\footnote{Основные его произведения~--- это <<Апология Сократа>>, диалоги <<Менон>>, <<Пир>>, <<Федр>>, <<Парменид>>, <<Государство>>, <<Законы>>.}.
	
	\sloppy Философия Платона носит характер \emph{объективного идеализма}. Истинные причины вещей Платон видит не~в~физической реальности, а~в~умопостигаемом мире, и~связывает их с~сущностями, которые он именует идеями (видами), или эйдосами (образами). Вещи материального мира могут меняться, они рождаются и~умирают, а вот их причины должны быть вечными и~неизменными, должны выражать ничем непоколебимую сущность вещей. Главный тезис Платона заключается в~том, что <<вещи можно видеть, но не~мыслить, идеи же, напротив, можно мыслить, но не~видеть>>.
	
	
	\section{Учения о~бытии и~познании Аристотеля}
	
	Аристотель (384--322 до~н.~э.)~--- величайший мыслитель Древней Греции, ученик Платона, подверг критике его учение. В~отличие от~Платона, Аристотель считал, что подлинным бытием обладает не~общее, не~идея, не~число, а~конкретная единичная вещь.
	
	\section{Основные отличия учений Платона и~Аристотеля}
	
	По~мнению Платона, идеи представляют собой нечто всеобщее, в~отличие от~единичных вещей, они единственно достойны познания. 
	
	Аристотель же вводит четыре класса причин процессов движения, изменения и~развития, происходящих в мире: материальные, формальные, действующие (деятельность) и~целевые. Он различает три вида души: растительную, животную и~разумную.
	
	Относительно живых тел, самой жизни, Аристотель, в~отличие от~Платона, настаивает на~неразрывной связи души и~тела. Душа без тела существовать не~может, её наличие свидетельствует о~завершённости тела, осуществлённости возможности жизни.
	
\end{document}